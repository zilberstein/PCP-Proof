\documentclass{beamer}

% \usepackage{beamerthemesplit} // Activate for custom appearance

\title{Dinur's Proof of the PCP Theorem}
\author{Tiernan Garsys, Lucas Pe\~{n}a, and Noam Zilberstein}
\date{\today}

\begin{document}

\frame{\titlepage}

\section[Outline]{}
\frame{\tableofcontents}

\section{Problem Statement}
\subsection{The PCP Theorem}
\subsection{}
\section{Main Result}
\subsection{Previous Approaches}
\subsection{Dinur's Approach}
\section{Technical Overview of the Proof}
\subsection{Definitions}
\subsection{The PCP Theorem}
\section{Proof of Lemmas}
\frame
{
  \frametitle{The PCP Theorem}

  \begin{itemize}
  \item<1-> The class NP is equivalent to the set of languages of problems that can be decided by a probabilistically checkable proof using $O(\log n)$ random bits and $O(1)$ query bits.
  \item<2-> Let PCP$_{\epsilon}[r, q, a]$ be the class of languages with a PCP using $r$ random bits and $q$ queries that each return $a$-bit responses.  If $x\in L$, then the verifier will always accept, if $x\not\in L$ then the verifier will accept with probability at most $\epsilon$.   
  \item<3-> Dinur presents a new proof that NP = PCP$_\frac12[c\log n, q, 1]$ where $c$ and $q$ are constants. 
  \end{itemize}
}

\frame{
  \frametitle{Constraint Graph Definition}
  \begin{itemize}
  \item<1->Let $G=\langle\langle V, E\rangle, \Sigma, \mathcal{C}\rangle$ be a constraint graph (CG) where:
  \begin{itemize}
    \item<2->$\langle V, E\rangle$ is a directed graph
    \item<3-> $\Sigma$ is a constant size set of colors
    \item<4-> $\mathcal C =\{c_e : \Sigma^2\mapsto \{0,1\}\mid e\in E\}$ is a set of constraints 
  \end{itemize}
  \item<5-> $G$ is a YES instance of CG iff:
  \begin{itemize}
    \item<6-> $\exists \sigma: V\mapsto \Sigma$ such that $\forall (u,v)\in E,\; c_{(u,v)}(\sigma(u),\sigma(v)) = 1$
  \end{itemize}
  \item<7-> \textbf{Example:} 3-Coloring:
  \begin{itemize}
    \item<8-> $\Sigma = \{$RED, GREEN, BLUE$\}$
    \item<9-> $\forall (u,v)\in E,$
    $$c_{(u,v)}(\sigma(u), \sigma(v)) = \begin{cases}
      1, & \sigma(u) \neq \sigma(v) \\
      0, & \text{otherwise}
      \end{cases}$$
  \end{itemize}
  \end{itemize}
}
\frame{
  \frametitle{Hardness of Constraint Graphs}
  \begin{itemize}
    \item<1-> Determining if a constraint graph is satisfiable is NP-complete
    \begin{itemize}
      \item<2-> Reduction from $k$-Coloring
      \item<3-> CG $\subseteq$ NP. Coloring is the proof, verification complexity is linear in number of edges
    \end{itemize}
  \end{itemize}
}  

\frame{
\frametitle{Technical Overview}
\begin{itemize}
\item<1->Show that the following are equivalent:
\begin{itemize}
\item<2-> Property 1: NP $\subseteq$ PCP$_{1-\epsilon}[O(\log n), 2, 4]$
\item<3-> Property 2: For any language $L\in$ NP there is a polynomial time transformation $T$ from instances of $L$ to a constraint graphs on 16 colors $\mathcal G_{16}$ such that if $x\in L$ then $\mathcal G_{16}$ is satisfiable and if $x\notin L$ then at most $(1-\epsilon)|E|$ of the constraints are simultaneously satisfiable
\end{itemize}
\item<4->It will suffice to prove Property 2
\item<5->Boost the accuracy of the solution to show that NP$\subseteq$ PCP$_{(1-\epsilon)^c}[O(\log n), 8c, 1]$
\end{itemize}
}

\frame{
\frametitle{Regarding Expanders}
\begin{itemize}
\item<1-> Edge expansion property for a graph $G =\langle V, E\rangle$: $\varphi(G) =\min_{|S|\le\frac n2}\left\{\frac{E(S,\bar s)}{|S|}\right\}$ such that $S\subseteq V$.
\end{itemize}
}

\frame{
\frametitle{The Gap Amplification Lemma}
\begin{block}{Lemma: Gap Amplification}
There exists a constant $0<\alpha<1$, an alphabet $\Sigma$, and a polynomial time reduction mapping the CG $G=\langle\langle V, E\rangle, \Sigma, \mathcal{C}\rangle$ to $G=\langle\langle V', E'\rangle, \Sigma', \mathcal{C}'\rangle$ such that:
\begin{itemize}
\item<2->$|G'|$ is $\Theta(|G|)$
\item<3->$\Sigma' = \Sigma$
\item<4->If UNSAT$(G) = 0$ then UNSAT$(G') = 0$
\item<5->UNSAT$(G') \ge 2\min\{$UNSAT$(G), \alpha\}$
\end{itemize}
\end{block}

\begin{block}<6->{Step 1: Preprocessing}
Convert $G$ to a constant degree expander. Worsens UNSAT$(G)$ by a constant factor, blows up $|G|$ by a constant factor
\end{block}

\begin{block}<7->{Step 2: Power Step}
Assuming that the degree of $G$ is constant, we amplify UNSAT$(G)$ while blowing up $|G|$ and $|\Sigma|$ by a constant factor.
\end{block}
}

\frame{
\frametitle{The Gap Amplification Lemma}
\begin{block}<1->{Step 1A}
Convert $G$ in to a constant degree graph
\begin{itemize}
\item<2->Let $G_n$ be a family of expander graphs of degree $d$ and edge expansion at least $\varphi_0$
\item<3->$G$ is transformed as follows:
  \begin{itemize}
    \item<4->Each vertex $v\in V$ (with degree $d_v$) is replaced with expander graph $G_{d_v}$ and each edge incident on $v$ is assigned to a vertex in $G_{d_v}$.
    \item<5->All edges in $G_{d_v}$ have equality constraints. All other edges retain original constraints
    \item<6->Now, we have that $|V'| = \sum_{v\in V}d_v = 2|E|$ and $|E'|=frac{(d+1)|V'|}{2}=(d+1)|E|$, so $|G'|$ is $\Theta(|G|)$.
    \item<7->If UNSAT$(G)=0$, then $UNSAT(G')=0$ by assigning each vertex in $G_{d_v}$ to the color of $v$
    \item<8->Now, we need to show that if UNSAT$(G)\neq 0$, then UNSAT$(G')$ is not much smaller
  \end{itemize}
\end{itemize}
\end{block}
}

\frame{
\frametitle{The Gap Amplification Lemma}
\begin{block}<1->{Step 1A}
If UNSAT$(G)\neq 0$, then UNSAT$(G')$ is not much smaller
\begin{itemize}
\item<2->Let $\sigma': V'\mapsto \Sigma$ be the best coloring for $G'$
\item<3->We now obtain $\sigma : V\mapsto \Sigma$ where $\sigma(v)$ is the most popular color in $\{\sigma'(u)\mid u\in G_{d_v}\}$
\item<4->Let $\mu = $UNSAT$(G)$
\item<5->$B$ is the set of edges violated by $\sigma$ and $B'$ is the set of edges violated by $\sigma'$
\item<6->$S=\{v\in V'\mid \sigma'(v)$ is not the popular color$\}$
\end{itemize}
\end{block}
}

\frame{
\frametitle{The Gap Amplification Lemma}
\begin{block}<1->{Case 1: $|B'|\ge \frac{\mu|E|}{2}$}
UNSAT$(G') = \frac{|B'|}{|E'|}\ge\frac{\mu|E|}{2|E'|} = \frac{\mu}{2(d+1)} = \frac{\text{UNSAT}(G)}{2(d+1)}$
\end{block}

\begin{block}<2->{Case 2: $|S|\ge \frac{\mu|E|}{2}$}
\begin{itemize}
\item<3->Consider $v\in V$ and corresponding cloud $G_{d_v}$ in $G'$
\item<4->$S^v = $ vertices in $G_{d_v}$ that did not get the popular color
\item<5->$\forall a\in\Sigma$, $\Sigma_a^v=\{v\in\Sigma^v\mid \sigma'(v)=a\}$
\item<6->$|S_a^v|<\frac{d_v}{2}$
\item<7->From the expansion property, $|E(S_a^v, \bar{S_a^v})| \ge \varphi_0|S_a^v|$
\item<8->All edge constraints in $E(S_a^v, \bar{S_a^v})$ are violated!
\item<9->$|B'|\ge\frac{\sum|E(S_a^v,\bar{S_a^v})|}{2}\ge\frac{\varphi_0|S|}{2}\ge\frac{\mu\varphi_0}{4}|E|\ge\frac{\mu\varphi_0}{4(d+1)}|E'| = $UNSAT$(G)\frac{\varphi_0}{4(d=1)}$
\end{itemize}
\end{block}
{In both cases, since UNSAT$(G')$ is optimal, we have proven that UNSAT$(G) \le k\cdot$UNSAT$(G')$ for some constant $k$}
}

\frame{
\frametitle{The Gap Amplification Lemma}
\begin{block}<1->{Step 1B}
\begin{itemize}
\item<2->$G'$ is $(d+1)-regular$
\item<3->Superimpose $\~{d}$-regular expander on $|V'|$ nodes
\item<4->The new superimposed graph has the same vertex set as the original constraint graph, but its edges are the union of the 2 graphs ($G'$ and the expander)
\item<5->Add self-loops to each vertex to get $G''$
\item<6->Impose dummy constraints on each new edge
\item<7->$G''$ is still an expander with constant degree $d+2+\~{d}$
\item<8->
\end{itemize}
\end{block}
}


\end{document}
