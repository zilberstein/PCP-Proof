\documentclass{beamer}

% \usepackage{beamerthemesplit} // Activate for custom appearance

\title{Dinur's Proof of the PCP Theorem}
\author{Tiernan Garsys, Lucas Pe\~{n}a, and Noam Zilberstein}
\date{\today}

\begin{document}

\frame{\titlepage}

\section[Outline]{}
\frame{\tableofcontents}

\section{Problem Statement}
\subsection{The PCP Theorem}
\subsection{}
\section{Main Result}
\subsection{Previous Approaches}
\subsection{Dinur's Approach}
\section{Technical Overview of the Proof}
\subsection{Definitions}
\subsection{The PCP Theorem}
\section{Proof of Lemmas}
\frame
{
  \frametitle{The PCP Theorem}

  \begin{itemize}
  \item<1-> The class NP is equivalent to the set of languages of problems that can be decided by a probabilistically checkable proof using $O(\log n)$ random bits and $O(1)$ query bits.
  \item<2-> Let PCP$_{\epsilon}[r, q, a]$ be the class of languages with a PCP using $r$ random bits and $q$ queries that each return $a$-bit responses.  If $x\in L$, then the verifier will always accept, if $x\not\in L$ then the verifier will accept with probability at most $\epsilon$.   
  \item<3-> Dinur presents a new proof that NP = PCP$_\frac12[c\log n, q, 1]$ where $c$ and $q$ are constants. 
  \end{itemize}
}

\frame{
  \frametitle{Constraint Graph Definition}
  \begin{itemize}
  \item<1->Let $G=\langle\langle V, E\rangle, \Sigma, \mathcal{C}\rangle$ be a constraint graph (CG) where:
  \begin{itemize}
    \item<2->$\langle V, E\rangle$ is a directed graph
    \item<3-> $\Sigma$ is a constant size set of colors
    \item<4-> $\mathcal C =\{c_e : \Sigma^2\mapsto \{0,1\}\mid e\in E\}$ is a set of constraints 
  \end{itemize}
  \item<5-> $G$ is a YES instance of CG iff:
  \begin{itemize}
    \item<6-> $\exists \sigma: V\mapsto \Sigma$ such that $\forall (u,v)\in E,\; c_{(u,v)}(\sigma(u),\sigma(v)) = 1$
  \end{itemize}
  \item<7-> \textbf{Example:} 3-Coloring:
  \begin{itemize}
    \item<8-> $\Sigma = \{$RED, GREEN, BLUE$\}$
    \item<9-> $\forall (u,v)\in E,$
    $$c_{(u,v)}(\sigma(u), \sigma(v)) = \begin{cases}
      1, & \sigma(u) \neq \sigma(v) \\
      0, & \text{otherwise}
      \end{cases}$$
  \end{itemize}
  \end{itemize}
}
\frame{
  \frametitle{Hardness of Constraint Graphs}
  \begin{itemize}
    \item<1-> Determining if a constraint graph is satisfiable is NP-complete
    \begin{itemize}
      \item<2-> Reduction from $k$-Coloring
      \item<3-> CG $\subseteq$ NP. Coloring is the proof, verification complexity is linear in number of edges
    \end{itemize}
  \end{itemize}
}  

\frame{
\frametitle{Technical Overview}
\begin{itemize}
\item<1->Show that the following are equivalent:
\begin{itemize}
\item<2-> Property 1: NP $\subseteq$ PCP$_{1-\epsilon}[O(\log n), 2, 4]$
\item<3-> Property 2: For any language $L\in$ NP there is a polynomial time transformation $T$ from instances of $L$ to a constraint graphs on 16 colors $\mathcal G_{16}$ such that if $x\in L$ then $\mathcal G_{16}$ is satisfiable and if $x\notin L$ then at most $(1-\epsilon)|E|$ of the constraints are simultaneously satisfiable
\end{itemize}
\item<4->It will suffice to prove Property 2
\item<5->Boost the accuracy of the solution to show that NP$\subseteq$ PCP$_{(1-\epsilon)^c}[O(\log n), 8c, 1]$
\end{itemize}
}

\end{document}
